\chapter{Critical Care}

\section{Introduction}
Expensive vs Ubiquitous
% Admission to the intensive care unit (ICU) in hospitals is associated with risk of acute mortality given critical illness for those patients, which can include shock and end organ damage. During this period, patients are closely monitored in the intensive care setting with both invasive and noninvasive sensor data and frequent lab monitoring in order to effectively determine and prevent complications of critical illness.  Physicians often use risk scores designed to identify and predict patients’ risk of conditions such as shock: general physiologic scoring like shock index, or more specifically septic shock (qSOFA score) or cardiogenic shock (completing Fick’s formula for cardiac index, cardiac output, and stroke volume).  I will present results in detecting patients at high and low risk for common ICD diagnosis codes given 16 seconds of their waveform signals.  A CNN was trained on a dataset of over 10,000 ICU patients using as input 3 electrocardiogram leads, photoplethysmogram, arterial line blood pressure, and respiration waveforms.  Inference on a new patient produces a heat map over a semantic space of diseases indicating conditions for which they are at high and low risk.

\section{Dataset}
TODO

\subsection{ICD Codes}
TODO

\subsection{Inputs}
TODO

\section{Model Architecture}
TODO

\section{Effective Sensitivity}
TODO

\subsection{Definition}
TODO

\subsection{Results}
TODO

\section{Relative Risk}
TODO

\subsection{Definition}
TODO

\subsection{Results}
TODO

\section{Semantic Space Inference}
TODO

\subsection{Definition}
TODO

\subsection{Results}
TODO