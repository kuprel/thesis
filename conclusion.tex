\chapter{Conclusion}
In the future, ubiquitous sensors will constantly monitor your health and assess your risk for life threatening conditions.  Many life threatening conditions can be cured if treated early enough.  Further, many life threatening conditions are obvious to a trained medical expert.  We can leverage the recent advances in AI to distill some of this highly valuable medical expertise.  Consequently, it can be distributed widely, run 24/7, and cost little more than the cost of electricity that runs it.  In this thesis I argued that future is not so distant.  Specifically I showed results in detecting skin cancer from photos that any smartphone can take, and assessing risk for a wide range of heart and other internal organ conditions from waveform signals such as ECG and PPG, both of which a smartwatch can measure.